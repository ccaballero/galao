\documentclass[letter,12pt]{article}

\usepackage[T1]{fontenc}
\usepackage{lmodern}
\usepackage{textcomp}
\renewcommand*\familydefault{\sfdefault}

\usepackage[spanish]{babel}
\usepackage[utf8x]{inputenc}

\usepackage[pdftex]{graphicx}
\usepackage{pifont}
\usepackage[
pdfauthor={Carlos Eduardo Caballero Burgoa},%
pdftitle={cappuchino},%
colorlinks,%
citecolor=black,%
filecolor=black,%
linkcolor=black,%
%urlcolor=black
pdftex]{hyperref}

\usepackage{fancyhdr}
\usepackage{lastpage}
\pagestyle{fancy}

% Para la primera página
\fancypagestyle{plain}{
\fancyhead[l]{}
\fancyhead[r]{}
\fancyhead[c]{}
\renewcommand{\headrulewidth}{0.5pt}
\fancyfoot[l]{SCESI \\ Sociedad Científica de Estudiantes de Sistemas e Informática}
\fancyfoot[c]{}
\fancyfoot[r]{\thepage/\pageref{LastPage}}
\renewcommand{\footrulewidth}{0.5pt}}

% Para el resto de páginas
\lhead{Proyecto Cappuchino}
\chead{}
\renewcommand{\headrulewidth}{0.4pt}
\lfoot{SCESI \\ Sociedad Científica de Estudiantes de Sistemas e Informática \\ 
\url {http://www.scesi.org}}
\cfoot{}
\rfoot{\thepage/\pageref{LastPage}}
\renewcommand{\footrulewidth}{0.4pt}

\title{\bf Proyecto Cappuchino}
\author{Carlos Eduardo Caballero Burgoa}

\begin{document}
\maketitle
\begin{center}\url {http://www.scesi.org}\end{center}
\pagebreak

\tableofcontents
\pagebreak

\section{Introducción}
Como parte de las construcciones de software orientado a soluciones primigenias, este documento
trata los asuntos referentes al desarrollo de un sistema para la manipulación de horarios en
la facultad de ciencias y tecnología.

\section{Antecedentes}
Cada semestre en la facultad de ciencias y tecnología, se pasa por un proceso clásico: compra de
matricula, publicación de horarios, e inscripción en el websiss.

En semestres bajos e intermedios, el proceso de seleccionar las materias que uno va a tomar para
el semestre, posee un gran esfuerzo de análisis, para que todas estas materias no colisionen, y que
ademas (si se puede), posean características que le convengan al estudiante, según sus propios criterios
y disponibilidades de tiempo.

En este proceso, se toman en cuenta muchas cosas entre otras:

\begin{itemize}
\item Minimizar las colisiones entre horarios.
\item Preferencia por un grupo en especifico.
\item Reducción de los puentes entre clases.
\item Restringir segun los tiempos de disponibilidad que se posee.
\end{itemize}

Todo esto concluye a realizar de este proceso, un trabajo moroso y hasta cierto punto agobiante,
particularmente cuando se tienen muchas posibilidades.

\section{Definición del Problema}

Por lo mencionado anteriormente se define el problema como:

La amplia variedad de posibilidades (ya sean materias o grupos) en la facultad conlleva a un gasto
innecesario de tiempo en la planificación de un horario adecuado para cualquier estudiante.

\section{Objetivo General}
Desarrollar un sistema web que permita a un estudiante seleccionar del amplio conjunto de
posibilidades de horarios para un semestre, aquel que considere mas prudente, para agilizar
y viabilizar una solución mas consciente.

\section{Objetivos Específicos}
\begin{itemize}
\item Automatizar el volcado de información desde los servidores de la facultad hacia el mismo sistema.
\item Facilitar a un estudiante la selección de grupos para un semestre.
\item Administrar de forma fácil el registro y gestión de usuarios del sistema.
\end{itemize}

\section{Proceso de desarrollo}
A continuación se detallan los pasos a seguir:

\begin{enumerate}
\item Construcción de los modelos necesarios tal que estos puedan albergar la estructura de la
información que se tiene a disposición.
\item Creación de la base de datos que pueda albergar la información requerida.
\item Construcción de las interfaces necesarias para el uso fácil del sistema.
\item Construcción de los controladores necesarios para la validación y filtrado de la información
entrante.
\item Desarrollo de un script que facilite el proceso de volcado de información desde los archivos
PDF provistos por la facultad.
\item Poner el sitio construido a un servidor en producción (cappuchino.scesi.org).
\end{enumerate}

\section{Herramientas}
En el axiomático caso de que a alguien le importe lo que estamos haciendo, se utilizaran herramientas
que faciliten el intercambio de código, e ideas que puedan mejorar la solución propuesta, entre otras
estas son:

\begin{enumerate}
\item FOS GNU/Linux: Como parte de la evaluación y a modo de contribución a este proyecto.
\item Apache2: Para el despliegue del sistema construido.
\item Git: Como herramienta de versionamiento y a futuro publicación del código del sistema en
el repositorio de proyectos github (https://github.com/).
\item NetBeans: Como IDE de desarrollo para PHP5.
\end{enumerate}

\section{Justificación}
Construir un sistema que ayude a los estudiantes a ver las opciones que tiene a la hora de tomar
materias, agilizará el tiempo de decisión, ademas de permitirle tomar decisiones mas adecuadas.

Esto será muy útil en cuestiones de tiempo y compromiso, con lo que cada quien crea prudente a la
hora de tomarse en serio su educación superior.

\end{document}
